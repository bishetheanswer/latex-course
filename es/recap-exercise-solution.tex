\documentclass[12pt]{article}

\usepackage{url}

\title{Ten Secrets to Giving a Good Scientific Talk}
\author{You}

\begin{document}
\maketitle

\section{Introduction}

The text for this exercise is a significantly abridged, and slightly modified, version of the excellent article of the same name by Mark Schoeberl and Brian Toon:
\url{http://www.cgd.ucar.edu/cms/agu/scientific_talk.html}

\section{The Secrets}

I have compiled this personal list of ``Secrets'' from listening to effective and ineffective speakers. I don't pretend that this list is comprehensive --- I am sure there are things I have left out. But, my list probably covers about 90\% of what you need to know and do.

\begin{enumerate}

\item Prepare your material carefully and logically. Tell a story.

\item Practice your talk. There is no excuse for lack of preparation.

\item Don't put in too much material. Good speakers will have one or two central points and stick to that material.

\item Avoid equations. It is said that for every equation in your talk, the number of people who will understand it will be halved. That is, if we let $q$ be the number of equations in your talk and $n$ be the number of people who understand your talk, it holds that
\begin{equation}
n = \gamma \left( \frac{1}{2} \right)^q
\end{equation}
where $\gamma$ is a constant of proportionality.

\item Have only a few conclusion points. People can't remember more than a couple things from a talk especially if they are hearing many talks at large meetings.

\item Talk to the audience not to the screen. One of the most common problems I see is that the speaker will speak to the viewgraph screen.

\item Avoid making distracting sounds. Try to avoid ``Ummm'' or ``Ahhh'' between sentences.

\item Polish your graphics. Here is a list of hints for better graphics:

\begin{itemize}
\item Use large letters.

\item Keep the graphics simple. Don't show graphs you won't need.

\item Use color.

\end{itemize}

\item Be personable in taking questions.

\item Use humor if possible. I am always amazed how even a really lame joke will get a good laugh in a science talk.

\end{enumerate}

\end{document}
